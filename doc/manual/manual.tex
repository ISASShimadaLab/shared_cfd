\documentclass{jsarticle}
\begin{document}
%
%%title{{{
%\title{SHIMADA LAB. CODE\\MANUAL}
%\author{山中翔太}
%\maketitle
%
%\tableofcontents
%\newpage
%%}}}
%
%%tutorial {{{
%\part{チュートリアル}
%\section{はじめに}
%
%\newpage
%%}}}
%
%details{{{
\part{各機能の説明}
\section{ライブラリ生成アルゴリズムについて}%{{{
本プログラムは基本的に「ライブラリ生成プログラム」となる。
よって、各種パラメータ設定ファイルに加えて、流体計算の場合は"condition.f90"というファイルで直接初期条件と境界条件を設定しなければならない。これらのファイルは"raw"というサフィックス付きでディレクトリにコピーされる。

GUIインターフェースであるGUI.pyは、checkout.pyを動かすためのinputファイル(checkout.inp, checkout\_model.inp, checkout\_chem.inp)を生成し、checkout.pyを実行するプログラムであり、主動作はGUI.pyを用いた場合でも全てcheckout.pyが受け持つ。

checkout.pyはinputファイルに基づき、各種ライブラリを生成するプログラムである。基本的にライブラリの生成は、該当ファイルをディレクトリからコピーすることによって行い、ある特定のファイルを除いてはfortranソースファイル自体をcheckout.pyが編集することはない。コピーされるファイルは全て"store"のサブディレクトリ内に入っており、checkout.pyまたはユーザーによって編集が加えられるファイルには".raw."というサフィックスがつけられている。なお、checkout.py内でそのソースファイルの編集が終了する、つまりそれ以上ユーザーによる編集が不必要になったファイルからは自動的に".raw."サフィックスが取り除かれ、そのままコンパイルにかけられるようになる。
\subsection{ディレクトリ構造}
基本的に類似する機能を持つファイルがまとめられている。例えばオイラー陽解法とLU-SGS法は"time\_scheme"ディレクトリ内に"euler","lu-sgs"としてそれぞれ格納されている。また化学計算と流体計算は基本的に同一ライブラリを用いることになっている。以下、全てのディレクトリについて説明を行う。
%}}}

\section{流体コード}%{{{
checkout.pyから呼び出された"store/checkout/checkout\_flow.py"によって生成される。生成されたファイルは"checkout"ディレクトリ内に生成される。
\subsection{自動生成ファイル}%{{{
流体コードで自動生成されるファイルは、checkout.pyで選択されたディレクトリ内にある"部分的なファイル"をつなげることで生成される。自動生成されるファイル及びその"部分的なファイル"は以下のとおりである。
\begin{itemize}
\item main.f90...主プログラムがかかれたソースとなる。主ループもここで回される。以下の順にファイルの内容が追記されていく。
\begin{itemize}
\item main.top.f90..."program main"からモジュールのインポートまで。
\item main.head.f90...共通して利用する変数の定義。
\item main.variable.f90...それぞれのスキームで特有に利用される変数の定義。
\item main.part\_init.f90...変数の初期化。グリッドの読み込みやそのバイナリファイルの生成や幾何ヤコビアンの生成、熱力学ライブラリが必要とされる場合にはその初期化も行われる。
\item main.body.raw.f90...初期化。restart.binがある場合はそれを用い、ない場合は初期化サブルーチンを用いて初期化を行う。また主ループを回す変数も初期化される。part\_primitiveという文字列がファイル内に記されており、後述する文字列に置き換えられる。主ループの開始まで記述してある。
\item main.main.raw.f90, main.main.point\_implicit.f90...
主ループ内部を記述しており、時間スキームのディレクトリ内に格納されている。
part\_point\_implicit, part\_primitive, part\_mainという文字列がファイル内に記されており、それぞれ以下の文字列に置き換えられる
part\_point\_implicitに文字列が存在する場合main.main.point\_implicit.f90、存在しない場合main.main.raw.f90が使われる。
なお、point implicitなスキームを受け付けない時間積分法(ex.Dual Time Step)にはmain.main.point\_implicit.f90は存在せず、無理やり使おうとする場合コンパイルエラーが起きる。
\begin{itemize}
\item part\_point\_implicit...point implicitに計算される部分がここに入る。現状、化学生成項の時間積分を進めるサブルーチンがここに入りうる。main.part\_point\_implicit.f90にかかれた文字列に置き換えられる。
\item main.part\_primitive.f90...時間刻みの計算やデータアウトプット前にすべき処理、具体的には基本量の計算及び境界条件の設定をするサブルーチンがここに入る。main.part\_primitive.f90にかかれた文字列に置き換えられる。
\item main.part\_main.f90...高次精度化や対流項及び拡散項の計算、時間積分に必要な場合にはそれらの保存量ヤコビアンが計算される。main.part\_main.f90にかかれた文字列に置き換えられる。
\end{itemize}
\item main.foot.f90...主ループの終了及びMPI関数の終了処理、"end program main"を記述してある。
\end{itemize}
\item Makefile...Makefile. 以下の順にファイルの内容が追記されていく。
\begin{itemize}
\item Makefile.head...コンパイラに何を使うのか定義する。アーキテクチャ(PCかスーパーコンピュータ)によって変更される。
\item Makefile.var ...各種変数の定義
\item Makefile.main...各オブジェクトファイル作成ルール
\item Makefile.foot...プログラムの最終的なコンパイル部分及びcleanの定義
\end{itemize}
\item control.raw.inp...cflの定義やファイル出力間隔の定義など。使用する場合、MUSCLのパラメータもここで入力される。control.part.inpに記された内容が入力される。
\end{itemize}
%}}}
\subsection{store/core}%{{{
どのような場合にも使用されるファイルがまとめられてある。上記自動生成ファイルのための各種ファイルの他、以下のファイルが存在する。
\begin{itemize}
\item check\_convergence.f90...収束判定及び途中経過ファイル,restartファイルの生成を行っている。
\item init.vi...vim用初期化ファイル。これを読み込むと、RUNコマンドで./mainを走らせることができる。
\item inout.f90...restartファイル読み書きや途中経過ファイル書き込み。
\item n\_grid.raw.f90...重要なパラメータの定義。以下の文字列はcheckout.pyで置き換えられる。
\begin{itemize}
\item NPLANE...multi-blockでblockの数。muti-blockでない場合1.
\item NumI...それぞれの面でのi方向セル数
\item NumJ...それぞれの面でのj方向セル数
\item NIMAX...i方向セル数最大値
\item NJMAX...j方向セル数最大値
\item NumY...rhoの数.ex)理想気体:1, flame sheet,完全平衡モデル:2, 素反応モデル:化学種数
\item NV...拡散で空間微分をとるべき次元数.flame sheetで3,完全平衡モデルで化学種数.
\item GridFileName...plot3dで記述されたグリッドファイルの場所.
\end{itemize}
また以下の文字列はICやBCの設定に使える。
\begin{itemize}
\item dimq ... q(後述)の次元
\item dimw ... w(後述)の次元
\item indxg ... wに於ける比熱比のインデックス
\item indxht ... wに於ける質量あたり全エンタルピー(J/kg)のインデックス
\item indxR ... wに於ける質量あたり気体定数(J/kg/K)のインデックス
\item indxMu ... wに於ける粘性係数(Pa*s)のインデックス
\end{itemize}
\item prmtr.f90...piやモルあたりの気体定数"R\_uni"の定義.
\item read\_control.f90...control.inpの読み込み.全計算に共通する部分に限る。
\item scheme.f90...流束項評価サブルーチン。現在はSLAUのみ。
\item variable.f90...一般的に使われる変数がまとめられてある。
\begin{itemize}
\item q ... 保存量.
\begin{itemize}
\item インデックス1からnY : それぞれの気体グループ(理想気体なら全化学種、flame sheetか完全平衡モデルなら燃料と酸化剤、素反応モデルならそれぞれの化学種)の密度(kg/m\^3)
\item インデックスnY+1 : x方向運動量(kg*m/s)
\item インデックスnY+2 : y方向運動量(kg*m/s)
\item インデックスnY+3 : 全エネルギー(J/m\^3)
\end{itemize}
\item qp ... 前ステップのq
\item qpp ... 全ステップのqp
\item w ... 基本量
\begin{itemize}
\item インデックス1 : 全化学種の密度の和(kg/m\^3).
\item インデックス2 : x方向速度(m/s).
\item インデックス3 : y方向速度(m/s).
\item インデックス4 : 圧力(Pa)
\item インデックス5 ... nY+4 : それぞれの気体グループの質量分率
\item インデックスnY+5(=indxg @n\_grid.f90) : 比熱比
\item インデックスnY+6(=indxht@n\_grid.f90) : 全エンタルピー(J/kg)
\item インデックスnY+7(=indxR @n\_grid.f90) : 気体定数(J/kg/K)
\item インデックスnY+8(=indxMu@n\_grid.f90) : 粘性係数(Pa*s)
\end{itemize}
\item vhi ...それぞれの気体グループの生成熱を含む内部エンタルピー(J/kg)
\item wHli,wHri, wHlj, wHrj ... 高次精度化で予測されたw( ex. MUSCL).'l'は'left'、'r'は'right'、'i'はi方向に+1/2を足した位置、'j'はj方向に1/2を足した位置。例えば、wHli(:,2,3)は$w_{2+\frac 1 2,3}$の左側の値を表す。
\item TGi, TGj ... それぞれi方向j方向の流束項TG. TGの定義については嶋田テキストのSection 9参照. 
これらの値も1/2だけセル中心からずれた場所の値であり、ずれ方はwHと同じ。
\item TGvi, TGvj ... 粘性項.
\item Vol ... 軸対称問題ではそれぞれのセルの体積、二次元問題では面積。
\item dsi, dsj ... i方向j方向にそれぞれ垂直なセル辺の長さ。軸対称問題ではさらに半径がかけられる。これもセル中心から1/2だけずれた場所の値。
\item vni, vnj ... それぞれdsi, dsjで示された辺の直交正規ベクトル(絶対値1).向きはi,j正の向き。これもセル中心から1/2だけずれた場所の値。
\item xh, rh ... メッシュ格子点の座標.これもセル中心から1/2だけずれた場所の値。
\item x, r ... セル中心の座標.
\item dt\_mat ... local time stepのそれぞれのセルでの$\Delta t$
\item dt\_grbl ... global time stepの$\Delta t$.(スペルミスったので'r'です。)
\end{itemize}
\end{itemize}
また全ての計算において、以下の値をcontrol.inpで設定する必要がある。
\begin{itemize}
\item Max Step Number...計算を再開してからの最大ステップ数.
\item convergent RMS...収束とするエネルギー残渣.
\item CFL Number...Courant–Friedrichs–Lewy数.
\item File Output Period...ファイル出力周期.
\end{itemize}
%}}}
\subsection{store/architecture}%{{{
アーキテクチャ依存のサブルーチン.
サブディレクトリはPCとSCでそれぞれPCとスーパーコンピュータに関するファイルがある.
SCにはmod\_mpi.f90、PCにはmod\_mpi\_dummy.f90が使われる。
スーパーコンピュータの場合はgrid\_separation.inpからそれぞれのプロセッサが受け持つグリッドの範囲を計算し、変数を初期化する.
PCの場合はMPI関数のダミー関数を入れている。どちらの場合も以下のように受け持つブロックの範囲,i方向の範囲,j方向の範囲が定義される。
\begin{itemize}
\item ブロックの範囲...npsからnpe
\item ブロックplaneのi方向の範囲...nxs(plane)からnxe(plane)
\item ブロックplaneのj方向の範囲...nys(plane)からnye(plane)
\end{itemize}
パソコンの場合、nps=1, npe=Nplaneである。
%}}}
\subsection{store/dim}%{{{
軸対称流か二次元流かを選ぶものである。サブディレクトリは2dとq2dで、それぞれ以下のファイルが含まれている。
\begin{itemize}
\item store/dim/2d...geometry.f90とinit\_Sq.f90
\item store/dim/q2d...geometry.f90とSq.f90
\end{itemize}
Sqは軸対称流のとき式変形の際ソース項に出てくる値を定義しており、二次元流では0である。init\_Sq.f90とSq.f90はそれぞれSqに関する処理を行っている。
geometry.f90にはplot3dで記述されたグリッドファイルの読み込み及び可視化用バイナリファイルの生成、幾何ヤコビアンやセルの面積を計算するルーチンがまとめられている。
%}}}
\subsection{store/high\_order}%{{{
高次精度化を行うルーチンである。サブディレクトリはmusclとupwindであり、それぞれMUSCLスキームと風上差分が入っている。
MUSCLを用いる場合、以下の項目をcontrol.inpで設定する必要がある。
\begin{itemize}
\item MUSCL ON(1)/OFF(0)...MUSCLのon/off切り替え。1でon, 0でoff.その他の値は受け付けない。
\item MUSCL Precision Order...MUSCLのオーダー調整。2と3のみ受けつけ、2の場合は$\kappa =-1$, 3の場合は$\kappa=\frac 1 3$にMUSCLのパラメータが設定される。
\end{itemize}
%}}}
\subsection{store/time\_step}%{{{
global time stepまたはlocal time stepが設定される。ファイルはset\_dt.f90。またこのどちらかを選ぶことにより、checkout.pyのDT\_GLOBAL\_LOCALがdt\_mat(i,j,plane)またはdt\_grblが設定され、時間積分スキームの該当する部分で置き換えられる。
%}}}
\subsection{store/time\_scheme}%{{{
時間スキームの選択。main.main.raw.f90やmain.main.point\_implicit.f90,時間積分サブルーチン(time\_...f90)が入っており、主ループ(時間積分ループ)の中身を定義する。
\begin{itemize}
\item euler...オイラー陽解法.
\item RK2...ルンゲ=クッタ2次精度.
\item LU-SGS...LU-SGSで実装したオイラー陰解法.上記の他sch\_lusgs.f90, var\_lusgs.f90でLU-SGS法を行う。
\item NR...Newton-Raphson法.同様にLU-SGS法を使う.後述するパラメータを用いる.上記の他sch\_NR.f90, var\_NR.f90でLU-SGS法を行う。
\item precon...前処理法を用いたNewton-Raphson法.同様にLU-SGS法を使う.上記の他sch\_precon.f90, var\_precon.f90でLU-SGS法を行う。
仮時間ステップにも前処理行列をかけている.後述するパラメータを用いる.
\item dual...Dual Time Step.同様にLU-SGS法を使う.後述するパラメータを用いる.上記の他sch\_dual.f90, var\_dual.f90でLU-SGS法を行う。
\item preconLU-SGS...前処理法をオイラー陰解法(LU-SGS)で実装している。上記の他sch\_precon.f90, var\_precon.f90でLU-SGS法を行う。
\end{itemize}
なお、LU-SGSからpreconLU-SGSには"2d"と"q2d"というディレクトリがそれぞれあるが、それぞれ二次元流、軸対称流のコードが収められている。

\paragraph{NR,precon,dualで用いられるパラメータについて}%{{{
これらのスキームでは以下のパラメータをcontrol.inpで定義する。ただし、$\omega$は内部反復の緩和係数である。
\begin{itemize}
\item InternalLoop CFL tau...疑似時間ステップのcfl数
\item InternalLoop Omega Max...最大緩和係数。以下$\omega_{\max}$とする。
\item InternalLoop Omega Min...最小緩和係数.これを割り込む$\omega$が設定されると、計算を中止する。
\item InternalLoop Dqrate Max...各気体グループの密度の内部反復あたり変化割合の最大値.以下$Dq_{\max}$とする。
\item InternalLoop ResRateWarn...内部反復が収束していないことを標準出力に投げる最小エネルギー残渣比
\item InternalLoop ResRateErr...計算を中止する最小エネルギー残渣比
\item InternalLoop OutOmega...ファイルinternal\_res.datに緩和係数$\omega$の履歴を出すか。
\item InternalLoop Max Number...内部ループ数(固定)
\end{itemize}
これらのスキームの内部ループで、保存量$q$は緩和係数$\omega$をかけられた$\Delta q$により更新される。$\omega$は以下のように定義される。
\begin{equation}
\omega = \min(\omega_{\max} ,\frac{Dq_{\max}\rho_i}{\Delta \rho_i})
\end{equation}
ただし$i=1...$nYである。これにより、$\rho_i$の一内部反復での変化は$Dq_{\max}$以下に抑えられる。
この定義であると、$\omega$が小さくなりすぎることがある。それを補足するのが"最小緩和係数"のパラメータである。

内部ループの発散は'InternalLoop Omega Min', 'InternalLoop ResRateWarn', 'InternalLoop ResRateErr'で補足されることとなる。ここでエネルギー残渣比は(最後の内部反復でのエネルギー残渣)/(最初の内部反復でのエネルギー残渣)と定義している。なお、NR, precon, dualの時間スキームでは、基本コンセプトは"内部反復が収束"することに基づいているので、エネルギー残渣比が十分小さくならない場合はコンセプト自体が崩壊している。よって、その場合これらのスキームは使うべきではない。(これを見落としている研究は大変多く、私は悲しい。)

\subparagraph{パラメータ調整方法}
パラメータ調整には"internal\_res.dat"が使える。これは"InternalLoop OutOmega"を"true"にセットすることで得られる。
(このファイル出力は時間がかかるので、本計算のときには効率向上のため"false"にすべき。)
このファイルには1列目に内部反復数、2列目にDqrateにより計算された$\omega$($\omega_{\max}$で修正する前)、3列目にエネルギー残渣比が記録されている。
以下にこれらの出力を使ってパラメータを調節する手段の一つを記す。
\begin{itemize}
\item \textbf{'InternalLoop Max Number'を増やす}...内部反復数が小さいことが原因で、エネルギー残渣比が指数的に減少しているのに'InternalLoop ResRateErr'に達していないとき。
\item \textbf{'InternalLoop Max Number'を減らす}...'InternalLoop Max Number'に比べ遥かに少ない内部反復数で'InternalLoop ResRateErr'に達しているとき。
\item \textbf{'InternalLoop Omega Max'を増やす}...下記の目安に比べ内部反復の収束が遥かに遅く、$\omega_{\max}$により$\omega$が主に決まっているとき。
\item \textbf{'InternalLoop Omega Max'を減らす}...Dqrateにより決定された$\omega$('internal\_res.dat'の第二列目に記録された$\omega$)が指数的に増加せず、一回目の内部反復の$\omega$が$\omega _{\max}$により決定されている場合。
この状況は$\omega\Delta q$があまりにも大きすぎて、一回目の内部反復における$q$の推測値が悪くなりすぎ、そのためその後のステップで正しい値に修正できなくなった場合に起こる。
\item \textbf{'InternalLoop Dqrate Max'を増やす}...
内部反復の収束が遅く、$\omega$がDqrateによって主に決定されている場合。
\item \textbf{'InternalLoop Dqrate Max'を減らす}...
$\omega$が指数的に増加せず、一回目の内部反復でDqrateにより$\omega$が決定されている場合。
この状況も$\omega\Delta q$があまりにも大きすぎて、一回目の内部反復における$q$の推測値が悪くなりすぎ、そのためその後のステップで正しい値に修正できなくなった場合に起こる。
\end{itemize}
これらの手段は理論的なものではなく、私の経験によるものである。よって間違っている可能性もある。他の方法も試してください。

\subparagraph{パラメータの目安}
\begin{itemize}
\item \textbf{0.1 @ Omega Max}...もしこれで動く場合、0.8や0.9も可能な場合もある。Dual Time Stepでは0.01を使わざるを得なかった場合もある。
\item \textbf{0.001 @ Omega Min}...この値を下回り正しい解を出した計算に出会ったことはない。このパラメータは固定すべきであると思う。
\item \textbf{0.01 @ Dqrate Max}...現実問題このパラメータは多分子流にのみ使われうる。Dual Time Stepでは0.001を使わざるを得なかった場合もある。経験上0.1は大きすぎる。
\item \textbf{1.e-5 @ ResRateWarn, 1.e-3 @ ResRateErr}...経験上、エネルギー残渣比は3オーダーは下げなければ正しい解を出さない。よって1e-3は固定すべきだと考える。ResRateWarnは警告メッセージを出すだけであるので、好みの値を用いてよい。
\item \textbf{60 @ Max Number}...理想気体では経験上20で十分。多分子流をDual Time Stepで解いたとき、100必要だった場合もあった。
\end{itemize}
%}}}
%}}}
\subsection{store/viscosity}%{{{
non-viscousとviscousというディレクトリがあり、viscousにはその中にwith-nVとno-nV、さらにそれぞれに"2d"と"q2d"というディレクトリが収められている。
non-viscousが非粘性流、viscousが粘性流である。
2dとq2dの違いはtime\_schemeと同じ。

with-nVとno-nVについて、化学モデルには"単純に$\rho_i$の空間微分でエンタルピー拡散を扱えるもの"とそうでないものがあり、それぞれがwith-nVとno-nVに対応している。
with-nVの場合、"Yv"という"質量拡散によるエンタルピー拡散を考える際考慮しなければならない気体グループの質量分率"が熱力学ライブラリから生成される。
例えば、一段総括反応の場合、化学組成は"反応物としての"燃料と酸化剤の質量のみを追いかければいいが、質量拡散によるエンタルピー拡散を計算する場合"生成物としての"燃料、酸化剤、生成物それぞれの質量分率を計算せねばならず、with-nVが使用される。

ファイルはsch\_viscous.f90である。プラントル数Prとシュミット数Scはここで名前付き定数として共に1に定義してあるので、条件を変えたい場合はここを変更するとよい。
%}}}
\subsection{store/therm\_lib}%{{{
therm\_lib内には必ずthermal\_mode.f90が入っており、set\_thermo\_propというサブルーチンにより、前ステップに計算された
\begin{quotation}
圧力、気体定数、保存量
\end{quotation}
から
\begin{quotation}
温度の推定値、質量あたり内部エネルギー、各気体グループの質量分率
\end{quotation}
を計算し、そこから現ステップでの
\begin{quotation}
圧力、比熱比、質量あたり全エンタルピー、粘性係数、気体定数、各化学種のエンタルピーvhi、各化学種の修正生成熱DHi
\end{quotation}
を計算する。
また、viscosityの節でも述べたとおり、一部熱力学ライブラリではエンタルピー拡散の計算のためにYvも計算する。
(圧縮性流体ではエネルギー保存であるためエンタルピーは圧力依存となる。圧力は温度で変わり、温度は熱力学ライブラリで変わるため、全エンタルピーも基本量としている。)
これら出力結果は基本量wの配列に格納される。

化学ライブラリには大きくわけて理想気体,NASAデータベースを用いるもの,chemkinファイルを用いるものにわけられ、それぞれディレクトリideal,NASA,chemkinに対応する。
NASAデータベースを用いる計算は凍結流、一段総括反応、完全平衡モデルがつかえ、
chemkinファイルを用いる計算は凍結流、素反応モデルが使える。

\paragraph{vhiとDHiの定義}%{{{
それぞれのライブラリ詳細の説明を行う前に、vhiとDHiの説明をする。

vhiはエンタルピー拡散の計算に使われ、次元はnV(Yvの次元と同じ.Yvを使わない場合はnY)である。
それぞれの気体グループの生成エネルギーを含む単位質量あたりの内部エンタルピーである。
拡散の計算に使われるため、vhiに関しては境界での値が必要となる。よって、適切な境界条件を設定しなければならない。

DHiは修正生成熱であり、陰解法や前処理法に用いる流束の保存量でのヤコビアンを求めると出てくる。定義としては以下のとおり。
\begin{equation}
DHi_i=\kappa R_i T-(\kappa-1)vhi_i
\end{equation}
ただし、$\kappa$は混合ガス全体の平均比熱比、$R_i$はその気体グループの質量あたりの気体定数(=(一般気体定数)/(平均分子量)).

$DHi_i$は理想気体では0であるが、実在気体では有限の値を持つ。また、陰解法を許さない熱力学モデルでは、これは0にセットされている。
%}}}

\subsubsection{store/therm\_lib/ideal}%{{{
\begin{center}
\begin{tabular}{cccc}\hline
特有モジュール & nY & nV & with-nV or no-nV\\\hline
gas            &  1 &  1 & no-nV\\
\hline
\end{tabular}
\end{center}

理想気体。ファイルはthermal\_model.f90のみである。粘性係数は動粘性係数一定の場合にのみ現在対応している。質量あたり気体定数R\_gas、比熱比kappa\_gas、動粘性係数nu\_gasは"module gas"をインポートすれば使用可能であり、checkout.pyでそれぞれ1.4,287,1.6e-5に定義されるが、thermal\_model.f90の該当部分を直接編集すれば任意の値に変更可能である。境界条件設定時にはぜひ"module gas"をインポートして上記の名前付き定数を使っていただきたい。

保存量、基本量やvhiは以下のように表される。ただし、$\rho,u,v,p$は密度、x,y方向速度、圧力である。
\begin{verbatim}
q(1)=rho
q(2)=rho*u
q(3)=rho*v
q(4)=p/(kappa_gas-1d0)

w(1)=rho
w(2)=u
w(3)=v
w(4)=p
w(5)=1d0
w(indxg )=kappa_gas
w(indxht)=kappa_gas/(kappa_gas-1d0)*p/rho+0.5d0*(u**2+v**2)
w(indxR )=R_gas
w(indxMu)=rho*nu_gas

vhi(1)=kappa_gas/(kappa_gas-1d0)*p/rho
DHi(1)=0d0
\end{verbatim}
%}}}
\subsubsection{store/therm\_lib/NASA}%{{{

\begin{center}
\begin{tabular}{cccc}\hline
特有モジュール               & nY &            nV & with-nV or no-nV\\\hline
const\_chem, chem, chem\_var &  2 &  モデルによる & with-nV\\
\hline
\end{tabular}
\end{center}

NASA-CEA用に公開されたデータベースを用いた熱力学モデル集。凍結流と一段総括反応(flame sheet)、完全平衡モデルを用意している。

気体グループは2つであり、ライブラリでは一つ目に燃料、二つ目に酸化剤を当てている。

全てのモデルで使用される熱力学モデルはcore内に格納されており、以下のファイルがある。
\begin{itemize}
\item LU.f90...LU分解で線形連立方程式を解くモジュール.完全平衡モデルで利用されている。
\item thermo.inp...NASA-CEA内に格納されている熱力学データ.
\item trans.inp...NASA-CEA内に格納されている熱輸送データ.
\item func\_chem.f90...熱力学関数多項式の係数を返す関数ライブラリ.
\item mod\_chem.raw.f90...NASAデータベースを用いた計算に必要なモジュール.(上述特有モジュール)変数後述.
\item sub\_chem.f90...mod\_chem内の変数初期化やinputファイル読み込み、上述反応モデルの全てのコア部のサブルーチン.
\end{itemize}

\paragraph{mod\_chem.f90の変数について}
\subparagraph{const\_chem}定数と重要な値(化学種数)を格納している.
\begin{itemize}
\item ne...原子種数
\item max\_ns...最大化学種数
\item Ru...一般気体定数(prmtr.f90と同じ値ではあるが、化学ライブラリのみで0次元を行うときのために定義している。)
\item pst=1d5...標準状態の圧力
\item omega=0.5d0...一段総括反応モデルで用いられる緩和係数
\item eps=1d-9...収束計算に用いる微小量.基準値がこれ以下になればよい.
\item initial\_eps=1d-5...完全平衡モデルで係数行列の行列式が0にならないようにそれぞれの化学種量初期値に加える微小量
\item Y\_eps=1d-6...完全平衡モデルで使用。燃料または酸化剤の質量分率がこれ以下の場合、存在しえない原子種に関する拘束条件を計算から排除する.
\item TSIZE=1d-11...完全平衡モデルで使用。EとHを計算する際、物質量がこれ以下の場合その分子がもつエネルギーやエンタルピーは計算しない。
\item TTSIZE=1d-14...完全平衡モデルで使用。各原子の量に関する拘束条件を計算するとき、物質量がこれ以下の分子は計算にいれない。
\item ns...化学種数.
\item nt...粘性係数の計算に使われる化学種数。化学組成の計算をする際に使われる化学種の内、trans.inpに含まれる化学種の数となる。
\end{itemize}

\subparagraph{chem}計算のコアとなるデータ.
\begin{itemize}
\item num\_sctn...それぞれの化学種の熱力学関数多項式の区間の数.最大値6.
\item MW...それぞれの化学種の分子量
\item Trange...ぞれぞれの区間の始めとおわりの温度.例えば化学種iのj番目の区間はTrange(1,j,i)KからTrange(2,j,i)Kまで.
\item co...それぞれの区間の熱力学関数の多項式の係数.例えば化学種iのj番目の区間の多項式の係数はco(1:9,j,i).
\item SYM\_ELM...それぞれの原子種の記号。(HEやH,Oなど)
\item SYM\_SPC...それぞれの化学種の記号。(CO2やH2Oなど)
\item trans...輸送係数の多項式の係数。熱力学関数のcoに対応。
\item Trange\_trans...輸送係数の区間の数。熱力学関数のTrangeに対応。
\item species\_name\_trans...熱力学関数のSYM\_SPCに対応
\item num\_sctn\_trans...熱力学関数のnum\_sctnに対応
\item tr2th...輸送係数のインデックスを熱力学関数のインデックスの変換するもの。
例えばH2Oのインデックスが輸送係数ではi、熱力学係数がj(このときspecies\_name\_trans(i)とSYM\_SPC(j)は共に'H2O'である。)の場合、
tr2th(i)=jである。
\item Ac...それぞれの化学種に含まれる原子の数.例えば原子H,C,Oのインデックスがそれぞれ1,2,3、分子H2Oのインデックスが1の場合、Ac(1,1)=2,Ac(2,1)=0,Ac(3,1)=1である。
\end{itemize}

\subparagraph{chem\_var}境界条件で使うための燃料,酸化剤のそれぞれの各種値、及びそれぞれのセルの前ステップでの化学組成の記録。

\begin{itemize}
\item n\_save(max\_ns,nimax,njmax,Nplane)
\item qf(dimq),wf(dimw)
\item rhof,pf,Tf,Ef,Hf,MWf,kappaf,muf
\item Yvf(nV),vhif(nV)
\item b0f(ne+2)
\item nf(max\_ns)
\item nfini(max\_ns)
\item nef
\item elistf(ne+2)
\item nelistf(ne+2)
\item maskf(max\_ns)
\item maskbf(ne+2)
\item 
\item qo(dimq),wo(dimw)
\item rhoo,po,To,Eo,Ho,MWo,kappao,muo
\item Yvo(nV),vhio(nV)
\item b0o(ne+2)
\item no(max\_ns)
\item noini(max\_ns)
\item neo
\item elisto(ne+2)
\item nelisto(ne+2)
\item masko(max\_ns)
\item maskbo(ne+2)
\item 
\item model
\item np(max\_ns)
\item of
\end{itemize}

最後に、境界条件での入力に関して。例えば酸化剤のみのセルで条件を入力する場合、保存量、基本量やvhiは以下のように表される。ただし、$u,v$は密度、x,y方向速度、圧力である。
DHiについてはset\_thermo\_propで直接入力されるため、手作業で入力されることはない。
\begin{verbatim}
q(1)=0d0
q(2)=rhoo
q(3)=rhoo*u
q(4)=rhoo*v
q(5)=rhoo*Eo

w(1)=rhoo
w(2)=u
w(3)=v
w(4)=po
w(5)=0d0
w(6)=1d0
w(indxg )=kappao
w(indxht)=Ho+0.5d0*(u**2+v**2)
w(indxR )=R_uni/MWo
w(indxMu)=muo

vhi=vhio
\end{verbatim}
燃料では\verb|q(1)=rhof;q(2)=0d0;w(5)=1d0;w(6)=0d0|になる。
%}}}
%}}}

\subsection{store/cond}%{{{
core
no-nV
with-nV
%}}}
%}}}


\section{NASA熱力学データベース自動生成}
\section{化学コード}
%}}}

%各々に対応表(星取表)
%
%\newpage
%
%\appendix
%\section{サンプル集}
%\section{前処理法の導出}
%\section{Dual Time Stepスキームの導出}
%
\end{document}
